\documentclass[compress]{beamer}
\usepackage[utf8]{inputenc}
\usetheme{autumn}

\title{Autumn}
\subtitle{A Custom Beamer Theme \subtitleline}
\author{Robin Truax}
\institute{Stanford University}
\date{July 14th, 2022}

\begin{document}
\begin{frame}[plain]
\titlepage
\end{frame}

\begin{frame}{Table of Contents}
    \tableofcontents
\end{frame}

\section{Blocks}
\subsection{Three Types}
\begin{frame}{Example Blocks}
There are a few different types of blocks.
\begin{block}{Definition}
This is an ordinary \texttt{block}.
\end{block}
\begin{alertblock}{Theorem}
This is an \texttt{alertblock}. 
\end{alertblock}
\begin{exampleblock}{Example}
This is an \texttt{exampleblock}.
\end{exampleblock}
\end{frame}

\section{Lists}
\subsection{Bulleted and Numbered}
\begin{frame}{Itemized Lists}
There are different types of lists. There are nested bullet lists: 
\begin{itemize}
    \item Item
    \begin{list}{$\circ$}{}
        \item Subitem
        \item Subitem
    \end{list}
    \item Item
\end{itemize}
There are also lists enumerated with numbers: 
\begin{enumerate}
    \item Item
    \item Item
\end{enumerate}
\end{frame}

\section{Math Mode}
\begin{frame}{Math Mode}
Unlike in standard Beamer presentations, anything in math mode uses serif. This means that math mode is consistent between in-presentation displays and graphics generated by TikZiT, SageMath, etc, and is also more similar to standard \LaTeX.
\begin{equation*}
    x^2 + y^2 = z^2.
\end{equation*}
\end{frame}

\end{document}
